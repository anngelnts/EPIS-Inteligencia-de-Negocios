\section{Introducción} 
\vspace{12mm} %5mm vertical space
Actualmente las organizaciones sin importar cuál sea su giro, se encuentran en una incertidumbre
permanente, debido al impacto de la globalización en los mercadosi
, estos cambios constantes en el
mercado y en los competidores, impactan a la empresa que ha obtenido “buenos resultados” durante años,
por lo que se ve en la necesidad de tener que cambiar de manera radical sus procesos, productos y/ o
servicios, en general la manera de llevar la administración de las organizaciones, pues estos cambios
(tecnológicos, sociales, culturales, económicos, etc.) obligan a generar indicadores, que den un mayor grado
de certeza en la toma de decisiones y así reducir el riesgo de fallar, de perder inversión o de no ser
competitivo, pues son estos algunas de las principales áreas que deberán estar vigilándose periódicamenteii
.
Tradicionalmente la evaluación de un negocio se basa en su liquidez económica, esto solo considera los
aspectos financieros, sin considerar el entorno organizacional, es decir, comúnmente no se toma en cuenta si
los clientes se encuentran satisfechos con los productos o servicios que las empresas generan, más aun no se
toma en cuenta si los procesos internos son los más adecuados o si la empresa está aprendiendo y
desarrollándose en base a ese aprendizaje.\newline


En la práctica es común que no se puedan identificar con claridad los puntos fuertes y débiles de la empresa
para la que se labora, ya que no es fácil analizar una situación cuando se está inmerso en ella, de hecho
seguramente para los directivos y propietarios de una organización será muy difícil encontrarse frente al
dilema “cambiar o morir”, mayormente si durante mucho tiempo se ha venido actuando de una manera
rutinaria, pero al confrontar situaciones relacionadas con: sacrificar margen de ganancia, perdida de ventaja
competitiva, crisis en el modelo estructural del negocio, por citar solo algunos puntos críticos para las
empresas, por esto se vuelve apremiante la necesidad de encontrar la manera de evaluar el desempeño de las
empresas.\newline
Este desempeño no puede medirse de la misma manera en todas las empresas aunque pertenezcan al
mismo sector como lo menciona Rouquette (2014) en su investigación de Análisis de varianza.
Por lo anterior es preciso tener claro qué es y para qué sirve la estrategia, de este modo “La estrategia
consiste en hacer un profundo análisis tanto de la organización como del entorno para definir un plan de
acción que lleve a mejorar la posición sobre los competidores”iv por otra parte la existencia de diferentes
herramientas que puedan complementar y mejorar estudios encaminados a validar la factibilidad de
emprendimiento. \newline


\vspace{16mm} %5mm vertical space