\section{Conclusiones}

En conclusión el balanced scorecard, es una forma de trabajar, es una metodología, con la cual el proceso de gestión se simplifica, nos permite tener una visión mas amplia de la empresa a futuro y de igual forma nos mantiene enfocado a cada uno de las partes de la organización en una sola dirección.\newline

A lo largo de este documento hemos visto como el balanced scorecard ayuda a entender mejor la estrategia, y enfocarnos en una sola visón, nos ayuda a simplificar esta tarea.
Siempre recordando que para lograrlo se necesita del apoyo de todo el personal de la organización y especialmente el apoyo de los máximos responsables, el seguimiento continúo de los indicadores para así asegurar el rumbo y avance de la organización.
Implantar el balanced scorecard, requiere de todo esto y mas recordando que el balanced scorecard no tiene una fin definido, si no es mas bien un ciclo, o como ya es muy conocido por nosotros los ingenieros industriales es una herramienta de mejora continua en toda la organización.
