\section{Abstract}

Business Intelligence delivers a rich set of benefits that drive significant and tangible return on investment. It removes the complexity of converting raw data into meaningful business intelligence by giving organizations the power to transform data from multiple sources into accurate, consumable information that can be shared securely throughout the enterprise. It enables users to make informed business decisions quickly and confidently by providing the query and reporting tools they need to find, share, manage, publish and analyze information. The goal of Business Intelligence is to enable management to make more intelligent decisions on the basis of knowledge extracted from data. Does this mean that having data is always good, that having more data and extracting more Knowledge from it is better, and that knowledge can be derived only from data?\newline

The paper also aims at describing processes of building Business Intelligence (BI) systems. Taking the BI systems specifics into consideration, the author presents a suggested methodology for the systems creation and implementation in organizations. The considerations are focused on the objectives and functional areas of the BI in organizations. Hence, in this context the approach to be used while building and implementing the BI involves two major stages that are of interactive nature, i.e. BI creation and BI “consumption”. A large part of the article is devoted to presenting Objectives and tasks that are realized while building and implementing BI.\newline

\textbf{Key words:} Business Intelligence, business decision-making, analytics, memory, monitoring.