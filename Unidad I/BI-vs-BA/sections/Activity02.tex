\section{Introducción} 
\vspace{12mm} %5mm vertical space
Muchas veces hemos escuchado estos términos de Business Intelligence y Business Analytics (BI y BA), y cuando nos preguntan sobre ellos no sabíamos diferenciarlos, llegando a pensar que podrían ser lo mismo.\newline

¿Qué es Business Intelligence? BI es un conjunto de técnicas que tiene por finalidad transformar los datos de una empresa en información, identificando posibles indicadores y que estos pueden ser explotados (por ejemplo, en cubos), ayudando a la mejora en la toma de decisiones.\newline

Hasta ahora vamos bien, pero y ¿Qué es Business Analytics? BA también es un conjunto de técnicas (algoritmos predictivos y modelos estadísticos), pero a diferencia de BI éstas nos permiten predecir posibles resultados (en un futuro), es aquí donde aparecen términos como: Machine Learning, Data Mining, etc.\newline

BI nos ayuda a responder preguntas como ¿Quién? ¿Cuándo paso? ¿Qué pasó?, mientras que BA nos ayuda a responder preguntas como ¿Cuándo volverá a suceder?  ¿Quiénes podrían volver a hacerlo?\newline

Si se enfoca en toda la data histórica hasta el presente, mostrándole por ejemplo en una empresa como han ido variando sus ventas a lo largo del o de los años, mientras que BA nos permitiría predecir en que mes (futuro) podríamos tener más ventas, teniendo como base los datos históricos que se obtuvieron de BI, es por eso que se puede concluir que estas dos técnicas no son ajenas la una de la otra, sino más bien son complementarias.\newline
\vspace{16mm} %5mm vertical space