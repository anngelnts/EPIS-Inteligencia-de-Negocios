%
% File acl2015.tex
%
% Contact: car@ir.hit.edu.cn, gdzhou@suda.edu.cn
%%
%% Based on the style files for ACL-2014, which were, in turn,
%% Based on the style files for ACL-2013, which were, in turn,
%% Based on the style files for ACL-2012, which were, in turn,
%% based on the style files for ACL-2011, which were, in turn, 
%% based on the style files for ACL-2010, which were, in turn, 
%% based on the style files for ACL-IJCNLP-2009, which were, in turn,
%% based on the style files for EACL-2009 and IJCNLP-2008...

%% Based on the style files for EACL 2006 by 
%%e.agirre@ehu.es or Sergi.Balari@uab.es
%% and that of ACL 08 by Joakim Nivre and Noah Smith

\documentclass[11pt]{article}
\usepackage{acl2015}
\usepackage{times}
\usepackage{url}
\usepackage{latexsym}
\usepackage[utf8]{inputenc}
\usepackage{hyperref}
%\setlength\titlebox{5cm}

% You can expand the titlebox if you need extra space
% to show all the authors. Please do not make the titlebox
% smaller than 5cm (the original size); we will check this
% in the camera-ready version and ask you to change it back.


\title{DATA STORYTELLING}

\author{First Author \\
    Layme Valeriano Diego \\
  {\tt email@domain} \\\And
  Second Author \\
    Moreno Mulluni Luis \\
    Mamani Calisaya Yonathan \\
  {\tt email@domain} \\}

\date{}

\begin{document}
\maketitle
\begin{abstract}
    \section{Resumen}
Tanto los datos como los almacenes de datos se utilizan de forma generalizada para almacenar big data, pero no hay términos intercambiables. Un lago de datos es un conjunto inmenso de datos en bruto. Un almacén de datos es un repositorio de datos filtrados y estructurados que ya han sido procesados ​​para una finalidad concreta.

La gente suele confundir estos tipos de almacenamiento de datos, cuando en realidad son mayores que sus semejanzas. A decir verdad, la única semejanza real entre ambos es su máxima finalidad, que es almacenar datos.

La diferencia es importante, porque están pensadas para los objetivos y los resultados. Mientras que a una empresa le conviene más tener un lago de datos, para obtener más resultados en la oportunidad de un almacén de datos.
\end{abstract}


\section{Introduction}

Ya sea que nos demos cuenta o no, nos bombardean con una avalancha de información visual todos los días. Desde anuncios impresos y comerciales de televisión hasta señales de alto y luces verdes, el mundo que nos rodea transmite un flujo constante de datos, a menudo acompañados de algún tipo de representación visual que nos ayuda a absorberlos rápidamente. Entonces, ¿cómo se comunican las empresas y las organizaciones con sus empleados, clientes y partes interesadas en la era de la gratificación instantánea de los datos? La respuesta es simple: narración de datos.}

\section{Objetivos}

Su objetivo es hacer que las personas se metan al interior de la historia, como en una buena película de cine, que nos lleva y nos embarca al interior. Esto sucede de forma similar en las presentaciones de las organizaciones, por eso cuando vemos en una presentación de diapositivas que pasan unas tras otras, que la gente se aburre, mira el reloj y hasta se queda dormida, es porque no les concierne; en la presentación no hay una historia. Por consiguiente el objetivo del Data Storytelling es que la presentación se convierta también tan cautivadora como una película, un libro o inclusive una obra de teatro. 

\section{Marco Teórico}

\large{DATA STORYTELLING}\\\\
La narración de datos es una metodología para comunicar información, adaptada a una audiencia específica, con una narrativa convincente. Son los últimos diez pies de su análisis de datos y posiblemente el aspecto más importante.


\subsection{Componentes principales}

\subsubsection{Datos}
Los datos están en todas partes, puedes extraerlos de censos, encuestas, sitios especializados, estudios hechos por alguna agencia de Marketing de Contenidos, publicaciones creadas por tu empresa, organismos gubernamentales, etcétera.

\begin{enumerate}
    \item \textbf{Originalidad:} los datos tienen que provenir de su fuente principal. Siempre busca la raíz de esa información. Antes de emplearla averigua, ¿quién la originó?, ¿es real, inventada o producto de una mala interpretación? Si no hay forma de descubrir el origen de una database y tampoco es posible chequear su veracidad, no la utilices.
    \item \textbf{Exhaustividad:} como te hemos contado en otros artículos de Postedin, uno de los objetivos que se persigue con el Marketing de Contenidos, es desarrollar publicaciones que sintonicen con los intereses y necesidades de sus consumidores. Entonces, la fuente debe ser lo suficientemente útil e informativa como para lograr ese propósito, lo que por supuesto, no quita la posibilidad de complementarla con otra, para crear el contenido adecuado.
\item \textbf{Actualidad:} a menos que necesites datos antiguos para contextualizar, siempre busca los más recientes. Recuerda que mucha información se desactualiza en un par de semanas e incluso, días. Así que procura que tu fuente aporte frescura a tu storytelling.
\item \textbf{Fiabilidad:} esta es una característica fundamental. Tus datos deben venir desde una fuente confiable y relevante. Tal como en el punto sobre originalidad, acá también es indispensable responder todas las preguntas necesarias antes de usarla en tu historia. ¿Qué autoridad tiene?, ¿existe el riesgo de incluir algún sesgo político?, ¿publica la metodología que emplea para recolectar los datos?, ¿los actualiza frecuentemente? No permitas que tu narración pierda calidad o credibilidad, por una mala elección.
\end{enumerate}


\subsubsection{Visualización}
Las visualizaciones le dan vida a sus datos y revelan ideas que solo las representaciones visuales sorprendentes y convincentes pueden lograr.

Debemos tener en cuenta Gestalt principles of visual perception:
\begin{enumerate}
    \item Proximity
    \item Similarity
    \item Enclosure
    \item Closure
    \item Continuity
    \item Connection
\end{enumerate}
\subsubsection{Narrativa}
Es cualquier técnica para crear imágenes, diagramas o animaciones para comunicar un mensaje.

\begin{enumerate}
            
        
        \item \textbf{Contexto}
			Lo primero que debe hacer es presentar la trama, construir el contexto para su audiencia. Considera esto el primer acto. En esta sección, configuramos los elementos esenciales de la historia: el entorno, el personaje principal, el estado de cosas no resuelto y el resultado deseado, lo que hace que todos se encuentren en un terreno común para que la historia pueda continuar. Deberíamos involucrar a nuestra audiencia, despertando su interés y respondiendo las preguntas que probablemente tengan en mente: ¿Por qué debo prestar atención? ¿Que hay para mi?
		\item \textbf{Reto}
			El desafió es la parte de una historia donde el héroe primero enfrenta el problema o la oportunidad. A menudo se le llama "complicación" o "catalizador" porque es el momento en el que una interrupción se lanza a los planes originales del héroe y desencadena toda la serie de eventos de la historia. En otras palabras, aquí es donde el héroe se encuentra con el villano.
		\item \textbf{Problema}
			El conflicto es donde el héroe lucha con el villano. Y como aprendimos, este es el corazón de una historia. Si tuviera que despojar todo lo demás de una historia, excepto el conflicto, sería interesante escucharla. De hecho, es la única parte de una historia que puede valerse por sí misma y aún dominar a una audiencia.
		\item \textbf{Desenlace}
			La resolución es donde explicas cómo resultó todo al final. ¿Ganó o perdió el héroe? ¿Funcionó el plan? ¿El villano fue atrapado o escapó? Es allí donde también puedes explicar cómo las cosas (incluidos los personajes) se cambiaron para siempre como resultado de la prueba.
		\item \textbf{Moraleja y Accion}
			La HISTORIA está técnicamente terminada en este punto, pero tu trabajo como narrador no lo es. Ahora es el momento de "salir" de la historia y hacer uso de ella, para darle sentido y dirigir la acción. Las tres cosas más productivas que puede hacer inmediatamente después de la historia son: (1) explicar la lección, (2) recomendar acción o (3) solo escuchar. Y usted podría hacer todos ellos.

\end{enumerate}
\subsection{¿Por qué Data Storytelling?}

\begin{enumerate}
	\item Comprensión rápida de la información: gracias a las representaciones gráficas podemos ver grandes cantidades de datos de forma clara y coherente, lo que facilita la extracción de conclusiones e insights. Ganaremos tiempo y eficiencia para solucionar problemas.
	\item Identificar y actuar rápido sobre tendencias emergentes: incluso los archivos de datos casi infinitos, empiezan a tener sentido al representarse gráficamente; lo que nos permite detectar parámetros que están altamente correlacionados. Algunas relaciones serán obvias, pero otras tendremos que identificarlas para ayudar al cliente a enfocarse en ese punto de mejora que influenciará en sus objetivos principales
	\item Identificar relaciones y patrones dentro de los activos digitales: descubrir tendencias dentro de los datos nos puede dar ventaja competitiva, como detectar puntos clave que están afectando a la calidad del producto o solucionar incidencias antes de que se conviertan en mayores problema
	\item Desarrollar un nuevo lenguaje de negocio para contar la historia a otros: una vez que hemos descubierto nuevos insights gracias a la analítica visual, el siguiente paso es comunicarlos, ya sea con gráficos simples o visualizaciones elaboradas, pero lo importante es lograr ese engagement y transmitir el mensaje rápidamente
\end{enumerate}




\section{Ejemplos}
\begin{enumerate}
    \item \textbf{Lead the Audience to Your Point}
    https://youtu.be/w9w08-NPivM
    \item \textbf{Data Stories Should Drive a Messag}
    https://youtu.be/bQlaKdXfKas
\end{enumerate}


\section{Análisis}

{\bf ¿A quién va dirigido el trabajo de realizar presentaciones con la nueva tendencia del Data Storytelling? }

Toda persona que requiera hacer presentaciones ya sea al interior o al exterior de la empresa deberá aprender las técnicas del Data Storytelling. Por ejemplo, el área financiera de las organizaciones, en donde encontramos a menudos tristes presentaciones, esta área es la que más requiere aprender sobre Data Storytelling para que lo que presenten sea interesante. Podemos citar otros muchos tipos de profesiones también, como los abogados, los médicos, los comerciales, en general todas las personas que tienen necesidad de comunicar algo, pueden aplicar esta nueva tendencia


{\bf ¿Qué competencias deben desarrollar los profesionales para convertirse en buenos «Data Storytellers? }
Podría mencionar tres competencias, la primera que sea conocedor del tema del cual va a hablar, la segunda que sea capaz de hacer buenos gráficos, el datavis, y de presentar visualmente los datos y la tercera es el de saber hablar en público, una especie de actor, es decir, que sepa ejercer el rol de presentador. Por otra parte, ya hemos comenzado a ver en algunas ofertas de empleo, que las organizaciones indican como criterio para contratar personal, conocimientos de las técnicas del Data Storytelling, como por ejemplo, una publicada hace un par de meses por la sociedad Nike, que buscaba una persona que se encargara del análisis de los medios sociales, un Data Storytelling Manager, precisando como requisito, entre otros, la capacidad de hacer el Data Storytelling. Un nuevo conocimiento complementario solicitado a los profesionales hoy día. 




\section*{Conclusiones}

Para persuadir al cliente lo mejor es afectar a sus emociones, y esto se consigue mediante la narrativa en la visualización de datos. Además nos asegura la atención de los receptores, pues toda historia bien construida con un principio y un desarrollo, hace que inconscientemente el público desee saber la conclusión del relato.

% include your own bib file like this:
%\bibliographystyle{acl}
%\bibliography{acl2015}


\begin{thebibliography}{9}
    \bibitem{Cole Nussbaumer Knaflic} 
    Cole Nussbaumer Knaflic
    \textit{Cole Nussbaumer Knaflic}. 
    Theory of Parsing, Translation and CompilingStorytelling with Data: A Data Visualization Guide for Business Professionals
     
    \bibitem{Mike Weinberg, Paul Smith} 
    Mike Weinberg, Paul Smith
    \textit{Sell with a Story}.
    Sell with a Story
     
    \bibitem{a} 
    Aprenda a narrar historias a partir de los datos con el «Data Storytelling»
    \\\texttt{www.decideo.com/Aprenda-a-narrar-historias-a-partir-de-los-datos-con-el-Data-Storytelling_a679.html}
    
    
        \bibitem{Nugit} 
    What is Data Storytelling?
    \\\texttt{www.nugit.co/what-is-data-storytelling/}
    
    \bibitem{analiticaweb} 
    Data Storytelling: qué necesitas saber
    \\\texttt{www.analiticaweb.es/data-storytelling-que-necesitas-saber/}
    
    \bibitem{Nugit} 
    What is Data Storytelling?
    \\\texttt{www.nugit.co/what-is-data-storytelling/}
    
    \bibitem{datahack} 
    Narrativa en la visualización de datos – La clave para transmitir nuestro Insight
    \\\texttt{www.datahack.es/data-storytelling}
    
    \bibitem{postedin} 
    DATA STORYTELLING: ¿CÓMO POTENCIAR LOS CONTENIDOS DE TU MARCA?
    \\\texttt{https://www.postedin.com/blog/2018/10/11/data-storytelling/}
    
    \bibitem{bi-notes} 
    Use these 7 Data Storytelling Examples to Make Your Insights more Meaningful Today
    \\\texttt{www.bi-notes.com/2018/06/examples-data-storytelling-analytics/}

    
    
    


\end{thebibliography}


\end{document}