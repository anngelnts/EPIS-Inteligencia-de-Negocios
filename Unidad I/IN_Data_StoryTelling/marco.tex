\section{MARCO TEORICO}

\begin{itemize}
    \item Data Base Management System (DBMS)\\
Sistema de administración de bases de datos. Software que controla la organización, almacenamiento, recuperación, seguridad e integridad de los datos en una base de datos. Acepta solicitudes de la aplicación y ordena al sistema operativo transferir los datos apropiados.

\item Base de datos\\

Una base de datos es un conjunto de datos pertenecientes a un mismo contexto y almacenados sistemáticamente para su posterior uso. En este sentido; una biblioteca puede considerarse una base de datos compuesta en su mayoría por documentos y textos impresos en papel e indexados para su consulta.

\item Seguridad\\

Ausencia de peligro o riesgo

\item Estrategia\\

una planificación de algo que se propone un individuo o grupo

\item Administrador de base de datos\\
Es aquel profesional que administra las tecnologías de la información y la comunicación, siendo responsable de los aspectos técnicos, tecnológicos, científicos, inteligencia de negocios y legales de bases de datos, y de la calidad de datos.


\item Sistema de prevención de intrusos\\

Un sistema de prevención de intrusos (o por sus siglas en inglés IPS) es un software que ejerce el control de acceso en una red informática para proteger a los sistemas computacionales de ataques y abusos. La tecnología de prevención de intrusos es considerada por algunos como una extensión de los sistemas de detección de intrusos (IDS), pero en realidad es otro tipo de control de acceso, más cercano a las tecnologías cortafuegos.


\item SQL\\
es un lenguaje específico del dominio utilizado en programación, diseñado para administrar, y recuperar información de sistemas de gestión de bases de datos relacionales
\end{itemize}
