\section{Resumen}
Tanto los datos como los almacenes de datos se utilizan de forma generalizada para almacenar big data, pero no hay términos intercambiables. Un lago de datos es un conjunto inmenso de datos en bruto. Un almacén de datos es un repositorio de datos filtrados y estructurados que ya han sido procesados ​​para una finalidad concreta.

La gente suele confundir estos tipos de almacenamiento de datos, cuando en realidad son mayores que sus semejanzas. A decir verdad, la única semejanza real entre ambos es su máxima finalidad, que es almacenar datos.

La diferencia es importante, porque están pensadas para los objetivos y los resultados. Mientras que a una empresa le conviene más tener un lago de datos, para obtener más resultados en la oportunidad de un almacén de datos.