\section{Discusión}

    \begin{enumerate}
        \item \textbf{Data lakes frente a almacenes de datos: ¿Cuál me conviene más?}
        
        \paragraph Las organizaciones suelen necesitar ambos. Los data lakes se crearon por la necesidad de sacar partido a big data y aprovechar los datos estructurados y no estructurados granulados sin procesar para el machine learning, pero sigue existiendo la necesidad de crear almacenes de datos para que los usuarios corporativos les den una aplicación analítica.
        
        \item \textbf{Sanidad: Los data lakes guardan información no estructurada}
        \paragraph{Los almacenes de datos llevan años utilizándose en el sector de la sanidad, pero jamás se han cosechado grandes éxitos. Debido a la naturaleza no estructurada de gran parte de los datos del sector sanitario (notas de facultativos, datos clínicos, etc.) y a la necesidad de obtener información útil en tiempo real, los almacenes de datos no suelen ser un modelo idóneo.

Los data lakes permiten una combinación de datos estructurados y no estructurados, lo que en general encaja mejor para las empresas de este sector.}
        
        \item \textbf{Educación: Los data lakes ofrecen soluciones flexibles}
        
        \paragraph{Los últimos años se ha puesto de manifiesto a todas luces el valor de big data en las reformas educativas. Los datos sobre las calificaciones de los alumnos, asistencia, etc., no solo pueden ayudar a los alumnos en apuros a volver a encauzar sus estudios, sino que pueden contribuir a predecir posibles problemas antes de que ocurran. Las soluciones flexibles de big data también han ayudado a los centro educativos a racionalizar su facturación, mejorar la recaudación de fondos y en muchos otros frentes.

Gran parte de estos datos son extensos y se encuentran totalmente sin procesar, de modo que a menudo a los centros de enseñanza les conviene más la flexibilidad de los data lakes.}
        
        \item \textbf{Finanzas: Los almacenes de datos atraen a las masas}
        \paragraph{En las finanzas, como en otros entornos de negocios, un almacén de datos suele ser el mejor modelo de almacenamiento, porque puede estructurarse de forma que toda la empresa tenga acceso y no estrictamente los científicos de datos.

Big data ha permitido que el sector de los servicios financieros dé pasos agigantados, y los almacenes de datos han tenido mucho que ver a ese progreso. El único motivo por el que una empresa de servicios financieros decida optar por otro modelo es porque, si bien resulta más rentable, no es tan eficaz para otras finalidades.}
        
        \item \textbf{Transporte: Los data lakes ayudan a realizar predicciones}
        
        \paragraph{La gran ventaja de la información que aporta un data lake pasa por la capacidad de realizar predicciones.

En el sector del transporte, en especial en la gestión de la cadena de suministros, la capacidad predictiva que surge de los datos flexibles en un data lake puede tener grandes ventajas, a saber, la posibilidad de rebajar los precios que aporta el análisis de los datos de formularios de la canalización de transporte.}

        
    \end{enumerate}